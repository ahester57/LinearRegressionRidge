
% CS 4710 HW 5

\documentclass{article}
\usepackage{titling}
\usepackage{siunitx}	% for scientific notation
\usepackage{graphicx}
\usepackage{caption}
\usepackage{animate}
\usepackage{listings}
\usepackage{microtype}

\setlength{\droptitle}{-9em}
\emergencystretch=1em
\oddsidemargin=12pt
\textwidth=450pt
\begin{document}

\title{CS 4340 - Project 5}
\author{Austin Hester}
\date{December 05, 2017}
\maketitle

% This makes it so sections aren't automatically numbered
\makeatletter
\def\@seccntformat#1{%
	 \expandafter\ifx\csname c@#1\endcsname\c@section\else
	  \csname the#1\endcsname\quad
  \fi}
\makeatother


%%%%%%%%%%%%%%%%%%%%%%%%%%%%%%%%%%

\section*{Introduction}

The goal of this project is to implement linear regression using validation and ridge regularization. \\

I did linear regression by matrix multiplication, so regularization does not seem to help much, if at all. \\

Our training data is:  

\begin{center}
\begin{tabular}{|c|c|}
	\hline
	\multicolumn{2}{|c|}{\textbf{Training Data}} \\\hline
	\textbf{X} & \textbf{y} \\\hline
	-2 & 14 \\
	-1 & 11\\
	0 & 10\\
	1 & 11 \\
	2 & 14\\
	3 & 19 \\
	4 & 26\\
	5 & 35\\
	6 & 46\\
	7 & 59\\
	8 & 74\\
	9 & 91\\
	10 & 110\\
	\hline
\end{tabular}
\end{center}

Our y is given by $$y = f(x) = x^2 + 10$$.

\newpage
\section*{Animated GIF of 3-Fold Validation}
\begin{center}
\hspace{-6em} \animategraphics[autoplay,controls,]{0.5}{./run}{0}{4}
\end{center}


%%%%%%%%%%%%%%%%%%%%%%%%%%%%%%%%%%
\newpage
\section*{The Code}
\lstset{
	frame=tb,
	tabsize=4,
	showstringspaces=false
}
\begin{lstlisting}[language=Python,breaklines=true]
# Austin Hester
# Linear Regression with Regularization
# CS 4340 - Intro to Machine Learning
# 12.04.17

import numpy as np
import random
import matplotlib.pyplot as plt

def make_x ():
    X = [ [ 1, i-2 ] for i in range ( 13 ) ]
    return np.array ( X )

def get_y ( X ):
    y = [ x**2 + 10. for x in X ]
    return np.array ( y )

def get_estimated_y ( X, w_lin ):
    return np.array ( [ round ( yi ) for yi in np.dot ( X, w_lin) ] )

def estimate_single ( N, w_lin ):
    return np.dot ( np.array ( [1, N] ), w_lin )

def print_results ( X, y, y_hat, w_lin ):
    print ( "X =\t", X )
    print ( "Wlin =\t", w_lin )
    print ( "y  =\t", y )
    print ( "y^ =\t", y_hat.T )
    slope, intercept = np.polyfit ( X, y_hat, 1 )
    print ( "Equation of linear regression line: ", "y = ",
            slope, "*x + ", intercept )
    
def plot ( X, y, y_hat, w_lin , turn=0 ):
    fig = plt.figure ( figsize=(6,6) )
    fig.suptitle ( "Linear Regression of Y on X" )
    # Plot training points
    ax = fig.add_subplot ( 1,1,1 )
    ax.scatter ( X, y, cmap='prism' )
    # Plot line learned from linear regression
    slope, intercept = np.polyfit ( X, y_hat, 1 )
    ablines = [ slope * i + intercept for i in X ]
    cx = fig.add_subplot ( 1,1,1 )
    cx.plot ( X, ablines, 'or-' )
    plt.ylabel ( 'y & y^' )
    plt.xlabel ( 'x' )
    plt.xlim ( -2, 10 )
    plt.ylim ( 0, 120 )
    #fig.savefig ( "run%d.png" % turn )

def get_MSE ( y, y_hat, w_norm, lam=1 ):
    total = 0
    for i in range ( y.size ):
        total += ( y [i] - y_hat [i] )**2
    return total / y.size + ( lam * w_norm )

def validate ( X, y, turn=0 , lam=1 ):

    if ( turn == 0 ):
        
        Xt, w_lin = linear_regression ( X [:8], y [:8])
        y_hat = get_estimated_y (X, w_lin)
        w_norm = np.dot ( w_lin.T, w_lin )
        print( "lambda = ", lam )
        print("MSE turn ", turn+1, " = ", get_MSE (y, y_hat, w_norm, lam ))
        return Xt, w_lin
    if ( turn == 1 ):
        middle = X[:4] + X[8:12]
        ymiddle = y[:4] +  y[8:12]
        Xt, w_lin = linear_regression ( middle, ymiddle )
        y_hat = get_estimated_y (X, w_lin)
        w_norm = np.dot ( w_lin.T, w_lin )
        print( "lambda = ", lam )
        print("MSE turn ", turn+1, " = ", get_MSE (y, y_hat, w_norm , lam ))
        return Xt, w_lin
    if ( turn == 2 ):
        Xt, w_lin = linear_regression ( X [4:], y [4:] )
        y_hat = get_estimated_y (X, w_lin)
        w_norm = np.dot ( w_lin.T, w_lin )
        print( "lambda = ", lam )
        print("MSE turn ", turn+1, " = ", get_MSE (y, y_hat, w_norm, lam ))
        return Xt, w_lin
    
def regularize_ridge ( X, y, y_hat, w_lin, lam=1 ):
    w_norm = np.dot ( w_lin.T, w_lin )
    validate ( X, y )
    return get_MSE ( y, y_hat, w_norm, lam )

def linear_regression ( X, y ):
    XxXT = np.dot ( X.T, X )            # Compute X.T * X <- A
    XxXT_inv = np.linalg.inv ( XxXT )   # Compute inverse of A <- B
    Xt = np.dot ( XxXT_inv, X.T )       # Psuedo-inverse of X = B * X.T <- C
    w_lin = np.dot ( Xt, y.T )          # Compute weight vector w_lin = C * y
    return Xt, w_lin.T

def run ():
    # Get training points
    X = make_x ()
    X1d = X.T [1] [:]
    y = get_y ( X1d )
    # Run linear regression on those points
    Xt, w_lin = linear_regression ( X, y )
    y_hat = get_estimated_y ( X, w_lin )
    plot ( X1d, y, y_hat , w_lin )
    # Regularize and Validate
    w_norm = np.dot ( w_lin.T, w_lin )
    print ( "MSE = ", get_MSE(y, y_hat, w_norm, 0.1 ))
    print_results ( X1d, y, y_hat, w_lin )
    for j in [ 0.1, 1, 10, 100 ]:
        for i in range(3):
            print ( "---------------------------------------------------" )
            print ( "Validation", i+1 )
            Xt, w_lin = validate ( X, y, i, j )
            print( w_lin )
            y_hat = get_estimated_y ( X, w_lin )
            plot ( X1d, y, y_hat , w_lin, i+1 )
            print_results ( X1d, y, y_hat, w_lin )


run ()
#plt.show ()


\end{lstlisting}
%%%%%%%%%%%%%%%%%%%%%%%%%%%%%%%%%%
\newpage
\section*{Notes}

\hspace{1em} \textbf{a)} Our training data.

\begin{center}
\begin{tabular}{|c|c|}
	\hline
	\multicolumn{2}{|c|}{\textbf{Training Data}} \\\hline
	\textbf{X} & \textbf{yl} \\\hline
	-2 & 14 \\
	-1 & 11\\
	0 & 10\\
	1 & 11 \\
	2 & 14\\
	3 & 19 \\
	4 & 26\\
	5 & 35\\
	6 & 46\\
	7 & 59\\
	8 & 74\\
	9 & 91\\
	10 & 110\\
	\hline
\end{tabular}
\end{center}

\textbf{b)} The equation of the line obtained in part (1) is: $$y = 8 x  + 8$$ 

\textbf{c)} For the following data: turn 1 is the first 8 points, turn 2 is all but the middle 4 points, and turn 3 is the last 8 points. 
I know there are 13 points total, but 13 is prime and we can ignore 1 here or there. \\

 $\lambda = 0.1$ \\
 MSE turn  1  =  746.8 \\
MSE turn  2  =  204.4 \\
MSE turn  3  =  550.7 \\

 $\lambda = 1$ \\
MSE turn  1  =  907.0 \\
MSE turn  2  =  451.0 \\
MSE turn  3  =  1016.7 \\

 $\lambda = 10$ \\
MSE turn  1  =  2509.0 \\
MSE turn  2  =  2917.0 \\
MSE turn  3  =  5676.7 \\

 $\lambda =100$ \\
MSE turn  1  =  18529.0 \\
MSE turn  2  =  27577.0 \\
MSE turn  3  =  52276.7 \\

\textbf{d)} I chose $\lambda$ = 0.1 because it gives the lowest MSE (besides zero). I do not really see the point of regularization 
when using linear regression. Linear regression is a non-iterative process, and it does not use MSE or SSE when obtaining a weight vector. \\

\textbf{e)} Our final equation of the line is: $$y = 7 x  + 15$$ \\


\subsection*{Example Run}
Here is example output of a run of 3-fold validation

\begin{lstlisting}[breaklines=true,basicstyle=\small]MSE =  166.8
X =	 [-2 -1  0  1  2  3  4  5  6  7  8  9 10]
Wlin =	 [ 8.  8.]
y  =	 [  14.   11.   10.   11.   14.   19.   26.   35.   46.   59.   74.   91.
  110.]
y^ =	 [ -8.  -0.   8.  16.  24.  32.  40.  48.  56.  64.  72.  80.  88.]
Equation of linear regression line:  y =  8.0 *x +  8.0
---------------------------------------------------
Validation 1
lambda =  0.1
MSE turn  1  =  746.8
[ 13.   3.]
X =	 [-2 -1  0  1  2  3  4  5  6  7  8  9 10]
Wlin =	 [ 13.   3.]
y  =	 [  14.   11.   10.   11.   14.   19.   26.   35.   46.   59.   74.   91.
  110.]
y^ =	 [  7.  10.  13.  16.  19.  22.  25.  28.  31.  34.  37.  40.  43.]
Equation of linear regression line:  y =  3.0 *x +  13.0
---------------------------------------------------
Validation 2
lambda =  0.1
MSE turn  2  =  204.4
[ 15.   7.]
X =	 [-2 -1  0  1  2  3  4  5  6  7  8  9 10]
Wlin =	 [ 15.   7.]
y  =	 [  14.   11.   10.   11.   14.   19.   26.   35.   46.   59.   74.   91.
  110.]
y^ =	 [  1.   8.  15.  22.  29.  36.  43.  50.  57.  64.  71.  78.  85.]
Equation of linear regression line:  y =  7.0 *x +  15.0
---------------------------------------------------
Validation 3
lambda =  0.1
MSE turn  3  =  550.777777778
[-19.33333333  12.        ]
X =	 [-2 -1  0  1  2  3  4  5  6  7  8  9 10]
Wlin =	 [-19.33333333  12.        ]
y  =	 [  14.   11.   10.   11.   14.   19.   26.   35.   46.   59.   74.   91.
  110.]
y^ =	 [ -43.  -31.  -19.   -7.    5.   17.   29.   41.   53.   65.   77.   89.
  101.]
Equation of linear regression line:  y =  12.0 *x +  -19.0
---------------------------------------------------
Validation 1
lambda =  1
MSE turn  1  =  907.0
[ 13.   3.]
X =	 [-2 -1  0  1  2  3  4  5  6  7  8  9 10]
Wlin =	 [ 13.   3.]
y  =	 [  14.   11.   10.   11.   14.   19.   26.   35.   46.   59.   74.   91.
  110.]
y^ =	 [  7.  10.  13.  16.  19.  22.  25.  28.  31.  34.  37.  40.  43.]
Equation of linear regression line:  y =  3.0 *x +  13.0
---------------------------------------------------
Validation 2
lambda =  1
MSE turn  2  =  451.0
[ 15.   7.]
X =	 [-2 -1  0  1  2  3  4  5  6  7  8  9 10]
Wlin =	 [ 15.   7.]
y  =	 [  14.   11.   10.   11.   14.   19.   26.   35.   46.   59.   74.   91.
  110.]
y^ =	 [  1.   8.  15.  22.  29.  36.  43.  50.  57.  64.  71.  78.  85.]
Equation of linear regression line:  y =  7.0 *x +  15.0
---------------------------------------------------
Validation 3
lambda =  1
MSE turn  3  =  1016.77777778
[-19.33333333  12.        ]
X =	 [-2 -1  0  1  2  3  4  5  6  7  8  9 10]
Wlin =	 [-19.33333333  12.        ]
y  =	 [  14.   11.   10.   11.   14.   19.   26.   35.   46.   59.   74.   91.
  110.]
y^ =	 [ -43.  -31.  -19.   -7.    5.   17.   29.   41.   53.   65.   77.   89.
  101.]
Equation of linear regression line:  y =  12.0 *x +  -19.0
---------------------------------------------------
Validation 1
lambda =  10
MSE turn  1  =  2509.0
[ 13.   3.]
X =	 [-2 -1  0  1  2  3  4  5  6  7  8  9 10]
Wlin =	 [ 13.   3.]
y  =	 [  14.   11.   10.   11.   14.   19.   26.   35.   46.   59.   74.   91.
  110.]
y^ =	 [  7.  10.  13.  16.  19.  22.  25.  28.  31.  34.  37.  40.  43.]
Equation of linear regression line:  y =  3.0 *x +  13.0
---------------------------------------------------
Validation 2
lambda =  10
MSE turn  2  =  2917.0
[ 15.   7.]
X =	 [-2 -1  0  1  2  3  4  5  6  7  8  9 10]
Wlin =	 [ 15.   7.]
y  =	 [  14.   11.   10.   11.   14.   19.   26.   35.   46.   59.   74.   91.
  110.]
y^ =	 [  1.   8.  15.  22.  29.  36.  43.  50.  57.  64.  71.  78.  85.]
Equation of linear regression line:  y =  7.0 *x +  15.0
---------------------------------------------------
Validation 3
lambda =  10
MSE turn  3  =  5676.77777778
[-19.33333333  12.        ]
X =	 [-2 -1  0  1  2  3  4  5  6  7  8  9 10]
Wlin =	 [-19.33333333  12.        ]
y  =	 [  14.   11.   10.   11.   14.   19.   26.   35.   46.   59.   74.   91.
  110.]
y^ =	 [ -43.  -31.  -19.   -7.    5.   17.   29.   41.   53.   65.   77.   89.
  101.]
Equation of linear regression line:  y =  12.0 *x +  -19.0
---------------------------------------------------
Validation 1
lambda =  100
MSE turn  1  =  18529.0
[ 13.   3.]
X =	 [-2 -1  0  1  2  3  4  5  6  7  8  9 10]
Wlin =	 [ 13.   3.]
y  =	 [  14.   11.   10.   11.   14.   19.   26.   35.   46.   59.   74.   91.
  110.]
y^ =	 [  7.  10.  13.  16.  19.  22.  25.  28.  31.  34.  37.  40.  43.]
Equation of linear regression line:  y =  3.0 *x +  13.0
---------------------------------------------------
Validation 2
lambda =  100
MSE turn  2  =  27577.0
[ 15.   7.]
X =	 [-2 -1  0  1  2  3  4  5  6  7  8  9 10]
Wlin =	 [ 15.   7.]
y  =	 [  14.   11.   10.   11.   14.   19.   26.   35.   46.   59.   74.   91.
  110.]
y^ =	 [  1.   8.  15.  22.  29.  36.  43.  50.  57.  64.  71.  78.  85.]
Equation of linear regression line:  y =  7.0 *x +  15.0
---------------------------------------------------
Validation 3
lambda =  100
MSE turn  3  =  52276.7777778
[-19.33333333  12.        ]
X =	 [-2 -1  0  1  2  3  4  5  6  7  8  9 10]
Wlin =	 [-19.33333333  12.        ]
y  =	 [  14.   11.   10.   11.   14.   19.   26.   35.   46.   59.   74.   91.
  110.]
y^ =	 [ -43.  -31.  -19.   -7.    5.   17.   29.   41.   53.   65.   77.   89.
  101.]
Equation of linear regression line:  y =  12.0 *x +  -19.0
\end{lstlisting}



\newpage


%%%%%%%%%%%%%%%%%%%%%%%%%%%%%%%%%%
\end{document}














